\documentclass[pdftex,12pt, twocol]{article} %[...,a4paper]
\usepackage[pdfborder={0,0,0}, hidelinks]{hyperref}
%\usepackage[pdfborder={0,0,0}]{hyperref}
\usepackage{makeidx}
\makeindex


\begin{document}
\begin{titlepage}
\begin{center}
		{\Huge \bf VCN Bots coding style}

(Copy-pasted and mainly based on Linux Kernel coding style)

(This document is under development. Date: 29/01/2016)
\end{center}
\end{titlepage}

\newpage
\tableofcontents

\newpage
\section{Introduction}
This is a short document describing the preferred coding style for VCN
Bots.  Coding style is very personal, and we won't \_force\_ our
views on anybody, but this is what goes for anything that we have to be
able to maintain, and we'd prefer it for most other things too.  Please
at least consider the points made here.

First off, we'd suggest printing out a copy of the GNU coding standards,
and NOT read it.  Burn them, it's a great symbolic gesture.

Anyway, here goes:

\section{Why is written in C (Standard 2011)?}
For many reasons. The main one is compability with all systems, arquitectures
and third-party APIs, such as MPI, OpenMP, Posix Threads, OpenCL, Cuda-C,
Java NDK, Python-C (Cython).

\section{Defensive programming}
The few basic rules of defensive programming are explained in a short chapter in Steve McConnell’s classic book on programming, Code Complete:
\begin{itemize}
\item Protect your code from invalid data coming from “outside”, wherever you decide “outside” is. Data from an external system or the user or a file, or any data from outside of the module/component. Establish “barricades” or “safe zones” or “trust boundaries” – everything outside of the boundary is dangerous, everything inside of the boundary is safe. In the barricade code, validate all input data: check all input parameters for the correct type, length, and range of values. Double check for limits and bounds.
\item After you have checked for bad data, decide how to handle it. Defensive Programming is NOT about swallowing errors or hiding bugs. It’s about deciding on the trade-off between robustness (keep running if there is a problem you can deal with) and correctness (never return inaccurate results). Choose a strategy to deal with bad data: return an error and stop right away (fast fail), return a neutral value, substitute data values, ...Make sure that the strategy is clear and consistent.
\item Don’t assume that a function call or method call outside of your code will work as advertised. Make sure that you understand and test error handling around external APIs and libraries.
\item Use assertions to document assumptions and to highlight “impossible” conditions, at least in development and testing. This is especially important in large systems that have been maintained by different people over time, or in high-reliability code.
\item Add diagnostic code, logging and tracing intelligently to help explain what’s going on at run-time, especially if you run into a problem.
\item Standardize error handling. Decide how to handle “normal errors” or “expected errors” and warnings, and do all of this consistently.
\item Use exception handling only when you need to, and make sure that you understand the VCN exception handler inside out.
\end{itemize}

Paranoid style:
\begin{verbatim}
 /* Bad */
 if (b == NULL) {}
 /* Good */
 if (NULL == bad) {}
\end{verbatim}


\section{Quasi-deprecated statements}

"register" key word is deprecated, because most compilers ignore it, and
store in register the approriate variables.

\section{Indentation}

Tabs are 8 characters, and thus indentations are also 8 characters.
There are heretic movements that try to make indentations 4 (or even 2!)
characters deep, and that is akin to trying to define the value of PI to
be 3.

Rationale: The whole idea behind indentation is to clearly define where
a block of control starts and ends.  Especially when you've been looking
at your screen for 20 straight hours, you'll find it a lot easier to see
how the indentation works if you have large indentations.

Now, some people will claim that having 8-character indentations makes
the code move too far to the right, and makes it hard to read on a
80-character terminal screen.  The answer to that is that if you need
more than 3 levels of indentation, you're screwed anyway, and should fix
your program.

In short, 8-char indents make things easier to read, and have the added
benefit of warning you when you're nesting your functions too deep.
Heed that warning.

The preferred way to ease multiple indentation levels in a switch statement is
to align the "switch" and its subordinate "case" labels in the same column
instead of "double-indenting" the "case" labels.  E.g.:

\begin{verbatim}
	switch (suffix) {
	case 'G':
	case 'g':
		mem <<= 30;
		break;
	case 'M':
	case 'm':
		mem <<= 20;
		break;
	case 'K':
	case 'k':
		mem <<= 10;
		/* fall through */
	default:
		break;
	}
\end{verbatim}

Don't put multiple statements on a single line unless you have
something to hide:

\begin{verbatim}
	if (condition) do_this;
	  do_something_everytime;
\end{verbatim}

Don't put multiple assignments on a single line either.  Kernel coding style
is super simple.  Avoid tricky expressions.

Outside of comments, documentation and except in Kconfig, spaces are never
used for indentation, and the above example is deliberately broken.

Get a decent editor and don't leave whitespace at the end of lines.


\section{Breaking long lines and strings}

Coding style is all about readability and maintainability using commonly
available tools.

The limit on the length of lines is 80 columns and this is a strongly
preferred limit.

Statements longer than 80 columns will be broken into sensible chunks, unless
exceeding 80 columns significantly increases readability and does not hide
information. Descendants are always substantially shorter than the parent and
are placed substantially to the right. The same applies to function headers
with a long argument list. However, never break user-visible strings such as
printk messages, because that breaks the ability to grep for them.


\section{Placing Braces and Spaces}

The other issue that always comes up in C styling is the placement of
braces.  Unlike the indent size, there are few technical reasons to
choose one placement strategy over the other, but the preferred way, as
shown to us by the prophets Kernighan and Ritchie, is to put the opening
brace last on the line, and put the closing brace first, thusly:

\begin{verbatim}
	if (x is true) {
		we do y
	}
\end{verbatim}

This applies to all non-function statement blocks (if, switch, for,
while, do).  E.g.:

\begin{verbatim}
	switch (action) {
	case KOBJ_ADD:
		return "add";
	case KOBJ_REMOVE:
		return "remove";
	case KOBJ_CHANGE:
		return "change";
	default:
		return NULL;
	}
\end{verbatim}

However, there is one special case, namely functions: they have the
opening brace at the beginning of the next line, thus:

\begin{verbatim}
	int function(int x)
	{
		body of function
	}
\end{verbatim}

Heretic people all over the world have claimed that this inconsistency
is ...  well ...  inconsistent, but all right-thinking people know that
(a) \verb+K&R+ are \_right\_ and (b) \verb+K&R+ are right.  Besides, functions are
special anyway (you can't nest them in C).

Note that the closing brace is empty on a line of its own, \_except\_ in
the cases where it is followed by a continuation of the same statement,
ie a "while" in a do-statement or an "else" in an if-statement, like
this:

\begin{verbatim}
	do {
		body of do-loop
	} while (condition);
\end{verbatim}

and

\begin{verbatim}
	if (x == y) {
		..
	} else if (x > y) {
		...
	} else {
		....
	}
\end{verbatim}

Rationale: \verb+K&R+.

Also, note that this brace-placement also minimizes the number of empty
(or almost empty) lines, without any loss of readability.  Thus, as the
supply of new-lines on your screen is not a renewable resource (think
25-line terminal screens here), you have more empty lines to put
comments on.

Do not unnecessarily use braces where a single statement will do.

\begin{verbatim}
	if (condition)
		action();
\end{verbatim}

and

\begin{verbatim}
	if (condition)
		do_this();
	else
		do_that();
\end{verbatim}

This does not apply if only one branch of a conditional statement is a single
statement; in the latter case use braces in both branches:

\begin{verbatim}
	if (condition) {
		do_this();
		do_that();
	} else {
		otherwise();
	}
\end{verbatim}

\subsection{Spaces}

Linux kernel style for use of spaces depends (mostly) on
function-versus-keyword usage.  Use a space after (most) keywords.  The
notable exceptions are sizeof, typeof, alignof, and \verb+__attribute__+, which look
somewhat like functions (and are usually used with parentheses in Linux,
although they are not required in the language, as in: "sizeof info" after
"struct fileinfo info;" is declared).

So use a space after these keywords:

\begin{verbatim}
	if, switch, case, for, do, while
\end{verbatim}

but not with sizeof, typeof, alignof, or \verb+__attribute__+.  E.g.,

\begin{verbatim}
	s = sizeof(struct file);
\end{verbatim}

Do not add spaces around (inside) parenthesized expressions.  This example is
*bad*:

\begin{verbatim}
	s = sizeof( struct file );
\end{verbatim}

When declaring pointer data or a function that returns a pointer type, the
preferred use of '*' is adjacent to the data name or function name and not
adjacent to the type name.  Examples:

\begin{verbatim}
	char *linux_banner;
	unsigned long long memparse(char *ptr, char **retptr);
	char *match_strdup(substring_t *s);
\end{verbatim}

Use one space around (on each side of) most binary and ternary operators,
such as any of these:

\begin{verbatim}
        =  +  -  <  >  *  /  %  |  &  ^  <=  >=  ==  !=  ?  :
\end{verbatim}

but no space after unary operators:

\begin{verbatim}
	&  *  +  -  ~  !  sizeof  typeof  alignof  __attribute__  defined
\end{verbatim}

no space before the postfix increment \verb+&+ decrement unary operators:

\begin{verbatim}
	++  --
\end{verbatim}

no space after the prefix increment \verb+&+ decrement unary operators:

\begin{verbatim}
	++  --
\end{verbatim}

and no space around the \verb+'.'+ and \verb+"->"+ structure member operators.

Do not leave trailing whitespace at the ends of lines.  Some editors with
"smart" indentation will insert whitespace at the beginning of new lines as
appropriate, so you can start typing the next line of code right away.
However, some such editors do not remove the whitespace if you end up not
putting a line of code there, such as if you leave a blank line.  As a result,
you end up with lines containing trailing whitespace.

Git will warn you about patches that introduce trailing whitespace, and can
optionally strip the trailing whitespace for you; however, if applying a series
of patches, this may make later patches in the series fail by changing their
context lines.


\section{Naming}

C is a Spartan language, and so should your naming be.  Unlike Modula-2
and Pascal programmers, C programmers do not use cute names like
ThisVariableIsATemporaryCounter.  A C programmer would call that
variable "tmp", which is much easier to write, and not the least more
difficult to understand.

HOWEVER, while mixed-case names are frowned upon, descriptive names for
global variables are a must.  To call a global function "foo" is a
shooting offense.

GLOBAL variables (to be used only if you \_really\_ need them) need to
have descriptive names, as do global functions.  If you have a function
that counts the number of active users, you should call that
"count\_active\_users()" or similar, you should \_not\_ call it "cntusr()".

Encoding the type of a function into the name (so-called Hungarian
notation) is brain damaged - the compiler knows the types anyway and can
check those, and it only confuses the programmer.  No wonder MicroSoft
makes buggy programs.

LOCAL variable names should be short, and to the point.  If you have
some random integer loop counter, it should probably be called "i".
Calling it "loop\_counter" is non-productive, if there is no chance of it
being mis-understood.  Similarly, "tmp" can be just about any type of
variable that is used to hold a temporary value.

If you are afraid to mix up your local variable names, you have another
problem, which is called the function-growth-hormone-imbalance syndrome.
See chapter 6 (Functions).


\section{Typedefs}

Please don't use things like \verb+"vps_t"+.
It's a \_mistake\_ to use typedef for structures and pointers. When you see a

\begin{verbatim}
	vps_t a;
\end{verbatim}

in the source, what does it mean?
In contrast, if it says

\begin{verbatim}
	struct virtual_container *a;
\end{verbatim}

you can actually tell what "a" is.

Lots of people think that typedefs "help readability". Not so. They are
useful only for:

\begin{itemize}
\item totally opaque objects (where the typedef is actively used to \_hide\_
     what the object is).

     Example: "pte\_t" etc. opaque objects that you can only access using
     the proper accessor functions.

     NOTE! Opaqueness and "accessor functions" are not good in themselves.
     The reason we have them for things like pte\_t etc. is that there
     really is absolutely \_zero\_ portably accessible information there.

\item Clear integer types, where the abstraction \_helps\_ avoid confusion
     whether it is "int" or "long".

     u8/u16/u32 are perfectly fine typedefs, although they fit into
     category (d) better than here.

     NOTE! Again - there needs to be a \_reason\_ for this. If something is
     "unsigned long", then there's no reason to do

	typedef unsigned long myflags\_t;

     but if there is a clear reason for why it under certain circumstances
     might be an "unsigned int" and under other configurations might be
     "unsigned long", then by all means go ahead and use a typedef.

\item when you use sparse to literally create a \_new\_ type for
     type-checking.

\item New types which are identical to standard C99 types, in certain
     exceptional circumstances.

     Although it would only take a short amount of time for the eyes and
     brain to become accustomed to the standard types like 'uint32\_t',
     some people object to their use anyway.

     Therefore, the Linux-specific 'u8/u16/u32/u64' types and their
     signed equivalents which are identical to standard types are
     permitted -- although they are not mandatory in new code of your
     own.

     When editing existing code which already uses one or the other set
     of types, you should conform to the existing choices in that code.

\item Types safe for use in userspace.

     In certain structures which are visible to userspace, we cannot
     require C99 types and cannot use the 'u32' form above. Thus, we
     use \_\_u32 and similar types in all structures which are shared
     with userspace.
\end{itemize}


Maybe there are other cases too, but the rule should basically be to NEVER
EVER use a typedef unless you can clearly match one of those rules.

In general, a pointer, or a struct that has elements that can reasonably
be directly accessed should \_never\_ be a typedef.


\section{Functions}

Functions should be short and sweet, and do just one thing.  They should
fit on one or two screenfuls of text (the ISO/ANSI screen size is 80x24,
as we all know), and do one thing and do that well.

The maximum length of a function is inversely proportional to the
complexity and indentation level of that function.  So, if you have a
conceptually simple function that is just one long (but simple)
case-statement, where you have to do lots of small things for a lot of
different cases, it's OK to have a longer function.

However, if you have a complex function, and you suspect that a
less-than-gifted first-year high-school student might not even
understand what the function is all about, you should adhere to the
maximum limits all the more closely.  Use helper functions with
descriptive names (you can ask the compiler to in-line them if you think
it's performance-critical, and it will probably do a better job of it
than you would have done).

Another measure of the function is the number of local variables.  They
shouldn't exceed 5-10, or you're doing something wrong.  Re-think the
function, and split it into smaller pieces.  A human brain can
generally easily keep track of about 7 different things, anything more
and it gets confused.  You know you're brilliant, but maybe you'd like
to understand what you did 2 weeks from now.

In source files, separate functions with one blank line.  If the function is
exported, the EXPORT* macro for it should follow immediately after the closing
function brace line.  E.g.:

\begin{verbatim}
	int system_is_up(void)
	{
		return system_state == SYSTEM_RUNNING;
	}
	EXPORT_SYMBOL(system_is_up);
\end{verbatim}

In function prototypes, include parameter names with their data types.
Although this is not required by the C language, it is preferred in Linux
because it is a simple way to add valuable information for the reader.


\section{Centralized exiting of functions}

Albeit deprecated by some people, the equivalent of the goto statement is
used frequently by compilers in form of the unconditional jump instruction.

The goto statement comes in handy when a function exits from multiple
locations and some common work such as cleanup has to be done.  If there is no
cleanup needed then just return directly.

Choose label names which say what the goto does or why the goto exists.  An
example of a good name could be \verb+"out_buffer:"+ if the \verb+goto+ frees "buffer".  Avoid
using GW-BASIC names like "err1:" and "err2:".  Also don't name them after the
goto location like \verb+"err_malloc_failed:"+

The rationale for using \verb+gotos+ is:
\begin{itemize}
\item Unconditional statements are easier to understand and follow.
\item Nesting is reduced.
\item Errors by not updating individual exit points when making modifications are prevented.
\item Saves the compiler work to optimize redundant code away.
\end{itemize}

\begin{verbatim}
	int fun(int a)
	{
		int result = 0;
		char *buffer;

		buffer = vcn_malloc(SIZE, GFP\_KERNEL);
		if (!buffer)
			return -ENOMEM;

		if (condition1) {
			while (loop1) {
				...
			}
			result = 1;
			goto out\_buffer;
		}
		...
	out\_buffer:
		vcn_free(buffer);
		return result;
	}
\end{verbatim}

A common type of bug to be aware of it "one err bugs" which look like this:

\begin{verbatim}
	err:
		vcn_free(foo->bar);
		vcn_free(foo);
		return ret;
\end{verbatim}

The bug in this code is that on some exit paths "foo" is NULL.  Normally the
fix for this is to split it up into two error labels "err\_bar:" and "err\_foo:".


\section{Commenting}

Comments are good, but there is also a danger of over-commenting.  NEVER
try to explain HOW your code works in a comment: it's much better to
write the code so that the \_working\_ is obvious, and it's a waste of
time to explain badly written code.

Generally, you want your comments to tell WHAT your code does, not HOW.
Also, try to avoid putting comments inside a function body: if the
function is so complex that you need to separately comment parts of it,
you should probably go back to chapter 6 for a while.  You can make
small comments to note or warn about something particularly clever (or
ugly), but try to avoid excess.  Instead, put the comments at the head
of the function, telling people what it does, and possibly WHY it does
it.

When commenting the kernel API functions, please use the kernel-doc format.
See the files Documentation/kernel-doc-nano-HOWTO.txt and scripts/kernel-doc
for details.

Linux style for comments is the C89 "/* ... */" style.
Don't use C99-style "// ..." comments.

The preferred style for long (multi-line) comments is:

\begin{verbatim}
	/*
	 * This is the preferred style for multi-line
	 * comments in the Linux kernel source code.
	 * Please use it consistently.
	 *
	 * Description:  A column of asterisks on the left side,
	 * with beginning and ending almost-blank lines.
	 */
\end{verbatim}

For files in net/ and drivers/net/ the preferred style for long (multi-line)
comments is a little different.

\begin{verbatim}
	/* The preferred comment style for files in net/ and drivers/net
	 * looks like this.
	 *
	 * It is nearly the same as the generally preferred comment style,
	 * but there is no initial almost-blank line.
	 */
\end{verbatim}

It's also important to comment data, whether they are basic types or derived
types.  To this end, use just one data declaration per line (no commas for
multiple data declarations).  This leaves you room for a small comment on each
item, explaining its use.


\section{You've made a mess of it}

That's OK, we all do.  You've probably been told by your long-time Unix
user helper that "GNU emacs" automatically formats the C sources for
you, and you've noticed that yes, it does do that, but the defaults it
uses are less than desirable (in fact, they are worse than random
typing - an infinite number of monkeys typing into GNU emacs would never
make a good program).

So, you can either get rid of GNU emacs, or change it to use saner
values.  To do the latter, you can stick the following in your .emacs file:

(defun c-lineup-arglist-tabs-only (ignored)
  "Line up argument lists by tabs, not spaces"
  (let* ((anchor (c-langelem-pos c-syntactic-element))
         (column (c-langelem-2nd-pos c-syntactic-element))
         (offset (- (1+ column) anchor))
         (steps (floor offset c-basic-offset)))
    (* (max steps 1)
       c-basic-offset)))

(add-hook 'c-mode-common-hook
          (lambda ()
            ;; Add kernel style
            (c-add-style
             "linux-tabs-only"
             '("linux" (c-offsets-alist
                        (arglist-cont-nonempty
                         c-lineup-gcc-asm-reg
                         c-lineup-arglist-tabs-only))))))

(add-hook 'c-mode-hook
          (lambda ()
            (let ((filename (buffer-file-name)))
              ;; Enable kernel mode for the appropriate files
              (when (and filename
                         (string-match (expand-file-name "~/src/linux-trees")
                                       filename))
                (setq indent-tabs-mode t)
                (setq show-trailing-whitespace t)
                (c-set-style "linux-tabs-only")))))

This will make emacs go better with the coding style for C.

But even if you fail in getting emacs to do sane formatting, not
everything is lost: use "indent".

Now, again, GNU indent has the same brain-dead settings that GNU emacs
has, which is why you need to give it a few command line options.
However, that's not too bad, because even the makers of GNU indent
recognize the authority of \verb+K&R+ (the GNU people aren't evil, they are
just severely misguided in this matter), so you just give indent the
options "-kr -i8" (stands for "\verb+K&R+, 8 character indents"), or use
"scripts/Lindent", which indents in the latest style.

"indent" has a lot of options, and especially when it comes to comment
re-formatting you may want to take a look at the man page.  But
remember: "indent" is not a fix for bad programming.

\section{Data structures}

Data structures that have visibility outside the single-threaded
environment they are created and destroyed in should always have
reference counts.  In the kernel, garbage collection doesn't exist (and
outside the kernel garbage collection is slow and inefficient), which
means that you absolutely \_have\_ to reference count all your uses.

Reference counting means that you can avoid locking, and allows multiple
users to have access to the data structure in parallel - and not having
to worry about the structure suddenly going away from under them just
because they slept or did something else for a while.

Note that locking is \_not\_ a replacement for reference counting.
Locking is used to keep data structures coherent, while reference
counting is a memory management technique.  Usually both are needed, and
they are not to be confused with each other.

Many data structures can indeed have two levels of reference counting,
when there are users of different "classes".  The subclass count counts
the number of subclass users, and decrements the global count just once
when the subclass count goes to zero.

Examples of this kind of "multi-level-reference-counting" can be found in
memory management ("struct mm\_struct": mm\_users and mm\_count), and in
filesystem code ("struct super\_block": s\_count and s\_active).

Remember: if another thread can find your data structure, and you don't
have a reference count on it, you almost certainly have a bug.


\section{Macros and Enums}

Names of macros defining constants and labels in enums are capitalized.

\begin{verbatim}
	#define CONSTANT 0x12345
\end{verbatim}

Enums are preferred when defining several related constants.

CAPITALIZED macro names are appreciated but macros resembling functions
may be named in lower case.

Generally, inline functions are preferable to macros resembling functions.

Macros with multiple statements should be enclosed in a do - while block:

\begin{verbatim}
	#define macrofun(a, b, c) 			\
		do {					\
			if (a == 5)			\
				do\_this(b, c);		\
		} while (0)
\end{verbatim}

Things to avoid when using macros:

\begin{enumerate}
\item macros that affect control flow:

\begin{verbatim}
	#define FOO(x)					\
		do {					\
			if (blah(x) < 0)		\
				return -EBUGGERED;	\
		} while(0)
\end{verbatim}

is a \_very\_ bad idea.  It looks like a function call but exits the "calling"
function; don't break the internal parsers of those who will read the code.

\item macros that depend on having a local variable with a magic name:

	\#define FOO(val) bar(index, val)

might look like a good thing, but it's confusing as hell when one reads the
code and it's prone to breakage from seemingly innocent changes.

\item macros with arguments that are used as l-values: FOO(x) = y; will
bite you if somebody e.g. turns FOO into an inline function.

\item forgetting about precedence: macros defining constants using expressions
must enclose the expression in parentheses. Beware of similar issues with
macros using parameters.

\begin{verbatim}
	#define CONSTANT 0x4000
	#define CONSTEXP (CONSTANT | 3)
\end{verbatim}

\item namespace collisions when defining local variables in macros resembling
functions:

\begin{verbatim}
#define FOO(x)				\
({					\
	typeof(x) ret;			\
	ret = calc\_ret(x);		\
	(ret);				\
})
\end{verbatim}

\verb+ret+ is a common name for a local variable - \_\_foo\_ret is less likely
to collide with an existing variable.
\end{enumerate}

\section{Printing messages}

VCN Bots developers like to be seen as literate. Do mind the spelling
of messages to make a good impression. Do not use crippled
words like "dont"; use "do not" or "don't" instead.  Make the messages
concise, clear, and unambiguous.

The messages do not have to be terminated with a period.

Printing numbers in parentheses (\%d) adds no value and should be avoided.

Coming up with good debugging messages can be quite a challenge; and once
you have them, they can be a huge help for remote troubleshooting.  However
debug message printing is handled differently than printing other non-debug
messages.  While the other pr\_XXX() functions print unconditionally,
pr\_debug() does not; it is compiled out by default, unless either DEBUG is
defined or CONFIG\_DYNAMIC\_DEBUG is set.  That is true for dev\_dbg() also,
and a related convention uses VERBOSE\_DEBUG to add dev\_vdbg() messages to
the ones already enabled by DEBUG.

\section{Allocating memory}

VCN Bots provides the following general purpose memory allocators:
\verb+vcn_malloc()+, \verb+vcn_calloc()+ and \verb+vc_realloc()+.

The preferred form for passing a size of a struct is the following:

\begin{verbatim}
	p = vcn_malloc(sizeof(*p), ...);
\end{verbatim}

The alternative form where struct name is spelled out hurts readability and
introduces an opportunity for a bug when the pointer variable type is changed
but the corresponding sizeof that is passed to a memory allocator is not.

Casting the return value which is a void pointer is redundant. The conversion
from void pointer to any other pointer type is guaranteed by the C programming
language.

The preferred form for allocating an array is the following:

\begin{verbatim}
	p = vcn_malloc\_array(n, sizeof(...), ...);
\end{verbatim}

The preferred form for allocating a zeroed array is the following:

\begin{verbatim}
	p = vcn_calloc(n, sizeof(...), ...);
\end{verbatim}

Both forms check for overflow on the allocation size n * sizeof(...),
and return NULL if that occurred.


\section{The inline disease}

There appears to be a common misperception that gcc has a magic "make me
faster" speedup option called "inline". While the use of inlines can be
appropriate (for example as a means of replacing macros), it
very often is not. Abundant use of the inline keyword leads to a much bigger
kernel, which in turn slows the system as a whole down, due to a bigger
icache footprint for the CPU and simply because there is less memory
available for the pagecache. Just think about it; a pagecache miss causes a
disk seek, which easily takes 5 milliseconds. There are a LOT of cpu cycles
that can go into these 5 milliseconds.

A reasonable rule of thumb is to not put inline at functions that have more
than 3 lines of code in them. An exception to this rule are the cases where
a parameter is known to be a compiletime constant, and as a result of this
constantness you *know* the compiler will be able to optimize most of your
function away at compile time. For a good example of this later case, see
the \verb+vcn_malloc()+ inline function.

Often people argue that adding inline to functions that are static and used
only once is always a win since there is no space tradeoff. While this is
technically correct, gcc is capable of inlining these automatically without
help, and the maintenance issue of removing the inline when a second user
appears outweighs the potential value of the hint that tells gcc to do
something it would have done anyway.


\section{Function return values and names}

Functions can return values of many different kinds, and one of the
most common is a value indicating whether the function succeeded or
failed.  Such a value can be represented as an error-code integer
(-Exxx = failure, 0 = success) or a "succeeded" boolean (0 = failure,
non-zero = success).

Mixing up these two sorts of representations is a fertile source of
difficult-to-find bugs.  If the C language included a strong distinction
between integers and booleans then the compiler would find these mistakes
for us... but it doesn't.  To help prevent such bugs, always follow this
convention:

	If the name of a function is an action or an imperative command,
	the function should return an error-code integer.  If the name
	is a predicate, the function should return a "succeeded" boolean.

For example, "add work" is a command, and the add\_work() function returns 0
for success or -EBUSY for failure.  In the same way, "PCI device present" is
a predicate, and the pci\_dev\_present() function returns 1 if it succeeds in
finding a matching device or 0 if it doesn't.

All EXPORTed functions must respect this convention, and so should all
public functions.  Private (static) functions need not, but it is
recommended that they do.

Functions whose return value is the actual result of a computation, rather
than an indication of whether the computation succeeded, are not subject to
this rule.  Generally they indicate failure by returning some out-of-range
result.  Typical examples would be functions that return pointers; they use
NULL or the ERR\_PTR mechanism to report failure.


\section{Don't re-invent the containers}

The header file include/vcn/container\_bot.h contains a number of macros that
you should use, rather than explicitly coding some variant of them yourself.
For example, if you need to calculate the length of an array, take advantage
of the macro

\begin{verbatim}
	#define ARRAY\_SIZE(x) (sizeof(x) / sizeof((x)[0]))
\end{verbatim}

Similarly, if you need to calculate the size of some structure member, use

\begin{verbatim}
	#define FIELD\_SIZEOF(t, f) (sizeof(((t*)0)->f))
\end{verbatim}

There are also min() and max() macros that do strict type checking if you
need them.  Feel free to peruse that header file to see what else is already
defined that you shouldn't reproduce in your code.


\section{Editor modelines and other cruft}

Some editors can interpret configuration information embedded in source files,
indicated with special markers.  For example, emacs interprets lines marked
like this:

	-*- mode: c -*-

Or like this:

\begin{verbatim}
	/*
	Local Variables:
	compile-command: "gcc -DMAGIC\_DEBUG\_FLAG foo.c"
	End:
	*/
\end{verbatim}

Vim interprets markers that look like this:

	/* vim:set sw=8 noet */

Do not include any of these in source files.  People have their own personal
editor configurations, and your source files should not override them.  This
includes markers for indentation and mode configuration.  People may use their
own custom mode, or may have some other magic method for making indentation
work correctly.

\section{Conditional Compilation}

Wherever possible, don't use preprocessor conditionals (\#if, \#ifdef) in .c
files; doing so makes code harder to read and logic harder to follow.  Instead,
use such conditionals in a header file defining functions for use in those .c
files, providing no-op stub versions in the \#else case, and then call those
functions unconditionally from .c files.  The compiler will avoid generating
any code for the stub calls, producing identical results, but the logic will
remain easy to follow.

Prefer to compile out entire functions, rather than portions of functions or
portions of expressions.  Rather than putting an ifdef in an expression, factor
out part or all of the expression into a separate helper function and apply the
conditional to that function.

If you have a function or variable which may potentially go unused in a
particular configuration, and the compiler would warn about its definition
going unused, mark the definition as \_\_maybe\_unused rather than wrapping it in
a preprocessor conditional.  (However, if a function or variable *always* goes
unused, delete it.)

Within code, where possible, use the IS\_ENABLED macro to convert a Kconfig
symbol into a C boolean expression, and use it in a normal C conditional:

\begin{verbatim}
	if (IS_ENABLED(CONFIG_SOMETHING)) {
		...
	}
\end{verbatim}

The compiler will constant-fold the conditional away, and include or exclude
the block of code just as with an \#ifdef, so this will not add any runtime
overhead.  However, this approach still allows the C compiler to see the code
inside the block, and check it for correctness (syntax, types, symbol
references, etc).  Thus, you still have to use an \#ifdef if the code inside the
block references symbols that will not exist if the condition is not met.

At the end of any non-trivial \#if or \#ifdef block (more than a few lines),
place a comment after the \#endif on the same line, noting the conditional
expression used.  For instance:

\begin{verbatim}
	#ifdef CONFIG_SOMETHING
	...
	#endif /* CONFIG_SOMETHING */
\end{verbatim}


\section*{Appendix I: References}
\addcontentsline{toc}{section}{Appendix I: References}
The C Programming Language, Second Edition
by Brian W. Kernighan and Dennis M. Ritchie.
Prentice Hall, Inc., 1988.
ISBN 0-13-110362-8 (paperback), 0-13-110370-9 (hardback).
URL: http://cm.bell-labs.com/cm/cs/cbook/

The Practice of Programming
by Brian W. Kernighan and Rob Pike.
Addison-Wesley, Inc., 1999.
ISBN 0-201-61586-X.
URL: http://cm.bell-labs.com/cm/cs/tpop/

GNU manuals - where in compliance with \verb+K&R+ and this text - for cpp, gcc,
gcc internals and indent, all available from http://www.gnu.org/manual/

WG14 is the international standardization working group for the programming
language C, URL: http://www.open-std.org/JTC1/SC22/WG14/

Kernel CodingStyle, by greg@kroah.com at OLS 2002:
http://www.kroah.com/linux/talks/ols\_2002\_kernel\_codingstyle\_talk/html/

\end{document}
